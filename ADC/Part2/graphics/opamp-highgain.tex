\documentclass{standalone}
\usepackage[utf8]{inputenc}
\usepackage{amsmath}
\usepackage[europeanresistors, europeanports]{circuitikz}
\usetikzlibrary{calc,positioning}

\begin{document}

\begin{circuitikz}
	\draw 
		coordinate (start)
        (start.center) to ++(0,-0.5) node[ground] {}
        (start.center) to[sV] ++(0,2)coordinate (source){}
		($(source)+(5mm,0)$) to[short,-o] ++(0,5mm) node[above=1mm]{$V_\text{in}$}
		
		node [op amp,right=50mm of source,anchor=-] (opamp) {}
		(opamp.-) -| ++(-5mm,0) node (node1){} node[below] {A}
		(opamp.out) to ++(0.5,0) coordinate (node3)
		
		%Feedback and gain path
		(node1.center) -- ++(0,15mm) to [R,l=$68\text k\Omega$] +(30mm,0) -| (node3)
		(source) -- ++(5mm,0) to[C=$1\mu$F] ++(20mm,0) to[R,l_=1.5k$\Omega$] ++(12mm,0) -- (node1.center) 
		
		%Reference voltage
		(opamp.+) -- ++(-10mm,0) node (node2) {}
		(node2.center) -- ++(0,10mm) to[R=10k$\Omega$,-o] ++(0,20mm) node[above] {$5\text{V}$}
		(node2.center) to[R,l_=10k$\Omega$] ++(0,-20mm) node[ground]{}

		(opamp.up) to[short,-o] ++(0,5mm) node[above] {$5\text{V}$}
		(opamp.down) to[short] ++(0,-5mm) node[ground] {}
		;

	\draw (node3) to[short,-o] ++(0.5,0) node[above] {$V_\text{out}$}
	;


	\draw node[right=4mm of opamp.down] {LT1366};
\end{circuitikz}

\end{document}
