\documentclass{standalone}
\usepackage[utf8]{inputenc}
\usepackage{amsmath}
\usepackage[europeanresistors, europeanports]{circuitikz}
\usepackage{adjustbox}
\usetikzlibrary{calc,positioning}

\begin{document}
\trimbox{-.25cm -.25cm -.25cm -.25cm}{ 
    \begin{circuitikz}[scale=1]
        \ctikzset{multipoles/thickness=3}
        \ctikzset{multipoles/dipchip/width=1}
        \def \ctikz@hook@start@draw@default {}
        
        \node [dipchip,num pins=6, hide numbers, no topmark,external pins width=0](state){REG2};
        \node [right] at (state.bpin 1) {D};
        \node [left] at (state.bpin 6) {Q};
        \draw (state.bpin 3) ++(0,0.2) -- ++(0.2,-0.2) -- ++(-0.2,-0.2);
        
        \ctikzset{multipoles/dipchip/width=1.5}
        \node [dipchip,num pins=6, hide numbers, no topmark,external pins width=0,left=of state](nlogic){NEXT};
        \node [right] at (nlogic.bpin 3) {IN};
        \node [right] at (nlogic.bpin 1) {Q};
        \node [left] at (nlogic.bpin 6) {N};
        
        \ctikzset{multipoles/dipchip/width=1.5}
        \node [dipchip,num pins=6, hide numbers, no topmark,external pins width=0,below=of nlogic,rounded corners](ologic){OUTPUT};
        \node [right] at (ologic.bpin 3) {IN};
        \node [right] at (ologic.bpin 1) {Q};
        \node [left] at (ologic.bpin 6) {Y};

        \draw (state.bpin 6) -| ++(1,1)
        -| ($(nlogic.bpin 1) + (-1,0)$) coordinate (csfork)
        -- (nlogic.bpin 1)
        (csfork) |- (ologic.bpin 1);

        \draw (nlogic.bpin 6) -- (state.bpin 1);

        \draw (nlogic.bpin 3) -- ++(-2,0) coordinate (infork)
        to[short,-o] ++(-1,0) node[left] {PB}
        (infork) |- (ologic.bpin 3);

        \draw (ologic.bpin 6) to[short,-o] ++(3,0) node[right] {Y};
        
    \end{circuitikz}
}

\end{document}
